% !TEX root = ../agfa-master.tex

%% %%%%%%%%%%%%%%%%%%%%%%%%%%%%%%%
%% ./AGFA-Einleitung.tex
%% Einleitung
%% Stand: 2022/10/10
%% %%%%%%%%%%%%%%%%%%%%%%%%%%%%%%%
\thispagestyle{empty}
\addsec{Einleitung}\label{sec:einleitung}
%
Dies ist die Einleitung zur Arbeit, manchmal auch Vorwort genannt.
%
\subsection*{Mein erster Unterabschnitt der Einleitung}
\subsubsection{}
Überall dieselbe alte Leier. 
Das Layout ist fertig, der Text lässt auf sich warten. 
Damit das Layout nun nicht nackt im Raume steht und sich klein und leer vorkommt, springe ich ein: der Blindtext. 
Genau zu diesem Zwecke erschaffen, immer im Schatten meines großen Bruders \enquote{Lorem Ipsum}, freue ich mich jedes Mal, wenn Sie ein paar Zeilen lesen. 

Diese Standards sorgen dafür, dass alle Beteiligten aus einer Webseite den größten Nutzen ziehen. Im Gegensatz zu früheren Webseiten müssen wir zum Beispiel nicht mehr zwei verschiedene Webseiten für den Internet Explorer und einen anderen Browser programmieren. Es reicht eine Seite, die - richtig angelegt - sowohl auf verschiedenen Browsern im Netz funktioniert, aber ebenso gut für den Ausdruck oder
\subsubsection{}
Denn esse est percipi - Sein ist wahrgenommen werden. Und weil Sie nun schon die Güte haben, mich ein paar weitere Sätze lang zu begleiten, möchte ich diese Gelegenheit nutzen, Ihnen nicht nur als Lückenfüller zu dienen, sondern auf etwas hinzuweisen, das es ebenso verdient wahrgenommen zu werden: Webstandards nämlich. Sehen Sie, Webstandards sind das Regelwerk, auf dem Webseiten aufbauen. So gibt es Regeln für HTML, CSS, JavaScript oder auch XML; Worte, die Sie vielleicht schon einmal von Ihrem Entwickler gehört haben.  
\subsubsection{}
Gehetzt sah er sich um. Plötzlich erblickte er den schmalen Durchgang. Blitzartig drehte er sich nach rechts und verschwand zwischen den beiden Gebäuden. 
Beinahe wäre er dabei über den umgestürzten Mülleimer gefallen, der mitten im Weg lag. 
Er versuchte, sich in der Dunkelheit seinen Weg zu ertasten und erstarrte: Anscheinend gab es keinen anderen Ausweg aus diesem kleinen Hof als den Durchgang, durch den er gekommen war. 
Die Schritte wurden lauter und lauter, er sah eine dunkle Gestalt um die Ecke biegen. 
Fieberhaft irrten seine Augen durch die nächtliche Dunkelheit und suchten einen Ausweg. 
War jetzt wirklich alles vorbei, \ldots
%%
\subsection*{Mein zweiter Unterabschnitt der Einleitung}
\subsubsection{}
Weit hinten, hinter den Wortbergen, fern der Länder Vokalien und Konsonantien leben die Blindtexte. 
Abgeschieden wohnen sie in Buchstabhausen an der Küste des Semantik, eines großen Sprachozeans. 
Ein kleines Bächlein namens Duden fließt durch ihren Ort und versorgt sie mit den nötigen Regelialien. 
\subsubsection{}
Es ist ein paradiesmatisches Land, in dem einem gebratene Satzteile in den Mund fliegen. 
Nicht einmal von der allmächtigen Interpunktion werden die Blindtexte beherrscht -- ein geradezu unorthographisches Leben. 
Eines Tages aber beschloß eine kleine Zeile Blindtext, ihr Name war Lorem Ipsum, hinaus zu gehen in die weite Grammatik. 
\subsubsection{}
Der große Oxmox riet ihr davon ab, da es dort wimmele von bösen Kommata, wilden Fragezeichen und hinterhältigen Semikoli, doch das Blindtextchen ließ sich nicht beirren. 
Es packte seine sieben Versalien, schob sich sein Initial in den Gürtel und machte sich auf den Weg. 
Als es die ersten Hügel des Kursivgebirges erklommen hatte, warf es einen letzten Blick zurück auf die Skyline seiner Heimatstadt Buchstabhausen, die Headline von Alphabetdorf und die Subline seiner eigenen Straße, der Zeilengasse. 
Wehmütig lief ihm eine rhetorische Frage über die Wange, dann setzte es seinen Weg fort. 
Unterwegs traf es eine Copy. Die Copy warnte das Blindtextchen, da, wo sie herkäme wäre sie%
\footnote{Weiteres findet man auf \href{https://www.blindtextgenerator.de}{https://www.blindtextgenerator.de}}
