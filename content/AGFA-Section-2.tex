% !TEX root = ../AGFA-Master.tex

%% %%%%%%%%%%%%%%%%%%%%%%%%%%%%%%
%% ./content/AGFA-Section-2.tex
%% Abschnitt 2 der Arbeit
%% Stand: 2022/10/10
%% %%%%%%%%%%%%%%%%%%%%%%%%%%%%%% 
\thispagestyle{empty}
\section{Mein zweiter Abschnitt}
\subsection{Test der Listen}\label{subsec:listen}
%%
\begin{myenumerate}
\item
Aufzählung
\item
Aufzählung
\end{myenumerate}
%%
\begin{myequivalent}
\item
Äquivalent
\item
Äquivalent
\end{myequivalent}
%%
\begin{myitemize}
\item
Punkte
\item
Punkte
\end{myitemize}
%%
\begin{mynumber}
\item
Nummeriert
\item
Nummeriert
\end{mynumber}
%%
\subsection{Test der mathematischen Umgebungen}\label{subsec:mumgebungen}
Schon seit vielen hundert Jahren eines der schönsten Ergebnisse der Mathematik.
%
\begin{theorem}\label{thm:theorem1}
In einem rechtwinkligen Dreiecke mit den Seiten $ a $, $ b $ und der Hypothenuse $ c $ gilt stets
\begin{equation}\label{eq:pythagoras}
a^{ 2 } +b^{ 2 } = c^{ 2 }
\end{equation}

$	a^{ 2 } +b^{ 2 } = c^{ 2 } $.
\end{theorem}
%
\begin{proof}
Für den Beweis verweisen wir auf die Literatur, etwa \textcite{efhn:2016}
\end{proof}
%%
\begin{corollary}\label{cor:folgerung}
Hieraus folgt dann 
%
\[
	a^{ 2 } + b^{ 2 } = c^{ 2 } \, .
\]
%
\end{corollary}
%
\begin{proposition}\label{prop:prop}
Und nun ein kleiner Satz als Ergänzung 
\end{proposition}
%
\begin{lemma}\label{lem:lemma}
Zuvor aber ein Lemma
\end{lemma}
%
\begin{remark}\label{rem:remark}
Eine Anmerkung
\end{remark}
%%
Mal sehen, wie die Eulersche Zahl und die imaginäre Einheit aussehen.
%%
\begin{theorem}\label{thm:eim}
$ \e^{ 2 \pi \im } = -1 $
\end{theorem}
%%
Alles andere kann jeder selbst mal testen.
%
\newpage
%%
\subsection{Querverweise}\label{subsec:referenzen}
\myquestion{Funktionieren alle Querverweise?}
%
Zunächst auf die Eulersche Zahl \vref{thm:eim} und dann auf Gleichung~\eqref{eq:pythagoras} in \vref{thm:theorem1}.

Oder auch möglich: \ldots \cref{thm:theorem1}, \vref{eq:pythagoras} \ldots.

Weitere Links: etwa auf \vref{lem:lemma} oder auf \vref{rem:remark} oder auf \vref{subsec:mumgebungen}.

Bitte in das \LaTeX-File reinsehen, wie dies gemacht ist.
%
\subsection{Sinnvolle Literatur zu \LaTeX}
%
Mal ansehen: \textcite{l2tabu} und \textcite{lshort-german} \bzw \textcite{latex-refsheet} für all die Befehle und Möglichkeiten.
Wie man sieht, sind die Links auf die Dokumente hinterlegt.

