% !TEX TS-program = pdflatexmk
% Wer nicht TeXShop nutzt kann die ersten beiden Zeilen löschen
%% %%%%%%%%%%%%%%%%%%%%%%%%%%%%%%%%%%%%%%%%%%%%%%%%%%%%%%%%%%%%%%%
%%  Thema:  AGFA-Master.tex  
% % Stand:	2022/03/08    
%% %%%%%%%%%%%%%%%%%%%%%%%%%%%%%%%%%%%%%%%%%%%%%%%%%%%%%%%%%%%%%%%
\documentclass[%	 -- siehe KOMA-Script 
	,toc		= bib 		 
	,parskip		= half-		 
	,headings	= normal			 	 		 
	,numbers		= noenddot		
	,leqno
	,version 	= last
	,DIV 		= calc
	,titlepage	= true
	%% -- Arbeit auf English oder Deutsch 
	%% entsprechend anpassen
	,ngerman				% Deutscher Text; 
%	,english				% oder Englischer Text	
	]{scrartcl}			% KOMA Artikelmodus
%% %%%%%%%%%%%%%%%%%%%%%%%%%%%%%%%%%%%%%%%%%%%%%%%%%%%%%%%%%%%%%%%

%% -- Hierin sind alle weiteren Pakete enthalten
%% -- Siehe agfa-readme.pdf für Details
%% -- 
 
\usepackage[%
	,thmframed			% gerahmte Umgebungen
	,numeric				% Nummeriertes LV
	,urldoi				% URL bzw. DOI unter dem Titel der Referenz
	%% -- Alternativer Zeichsatz
	%% -- Siehe ReadMe.pdf
%	,lmodern				% lmodern
%	,libertinus			% libertinus
	]{./preamble/agfa-art}  

%% -- Was soll aktuell bearbeitet werden
%% -- Mittels % entsprechend steuern
%% --

\includeonly{%
	./content/AGFA-Section-1,
	./content/AGFA-Section-2,
	}

%% -- Für den Druck % entfernen; 
%% -- Notwendige Bindekorrektur
%% --

% \KOMAoptions{BCOR = 12mm}

%% -- für die finale Version die folgenden Zeilen 
%% -- mit % auskommentieren
%% -- 

\KOMAoptions{footsepline}
\lofoot{Name}
\cofoot{Stand der Arbeit:}
\rofoot{\today}

%% -- Datei mit den Referenzen für Literatur
%% -- und Steuerung für url etc. 
%% -- siehe AGFA-ReadMe.pdf
%% --


\addbibresource{./bib/agfa-bib.bib}	
	
\ExecuteBibliographyOptions{%
	,backref		= true		% wo habe ich was referenziert, alles andere weg
 	,url		= false		% true	wenn online-Zitate separat gezeigt werden sollen
 	,doi		= false		% true  wenn DOI angezeigt werden sollen, sinnvoll bei Vorlesungen 
	,eprint		= false		% true  - ” -
	}

%% -- Querverweise
%% -- siehe agfa-readme.pdf
%% --

\hypersetup{
	,colorlinks	= true    			%  Farbige Links false/true, für onlineversion true                                                           
	,urlcolor	= blue       		%                                                              
	,citecolor	= blue      		    %                                                          
	,linkcolor	= blue			 	% oder black
%  	,hidelinks 						% Vor dem Druck % entfernen
	}
%% -- 

\begin{document}

%% %%%%%%%%%%%%%%%%%%%%%%%%%
%% Frontmatter
%% %%%%%%%%%%%%%%%%%%%%%%%%
%% -- kann man anfangs übergehen
%% -- daher %% bei \begin{comment} und \end{comment} entfernen/hinzufügen
%% --

%%\begin{comment}
\begingroup
	% !TEX root = ../AGFA-Master.tex
%% %%%%%%%%%%%%%%%%%%%%%%%%%%%%%%%
%% ./0-AGFA-title.tex
%% Titelseite
%% Stand: 2023/01/22
%% %%%%%%%%%%%%%%%%%%%%%%%%%%%%%%%
\begin{titlepage}
%%
\parbox{.75\linewidth}{%
Universität Tübingen \\ 
Mathematisch-Naturwissenschaftliche Fakultät \\ 
Fachbereich Mathematik} 
\par\vspace{3cm}
%%
\begin{center}\Large\bfseries
	\begin{tabular}{c}
		Bachelor/Masterarbeit \\[1cm]
		Der Titel der Arbeit \\
		-- Untertitel -- \\
	\end{tabular}\par\normalsize\normalfont
	\vspace*{2cm}%
%%
	\begin{tabular}{c}
		\large\textbf{Vorname Name} \\[20pt]
		Datum der Abgabe 		
	\end{tabular}\par\vspace*{\fill}
%%	
	\begin{tabular}[t]{l}
	Betreuer: \\[1ex]
	Prof.\,Dr. Vorname Name \\
	Prof.\,Dr. Vorname Name		
	\end{tabular}
\end{center}
%%
\end{titlepage}
\cleardoublepage			% Titelseite
	\pagestyle{empty}
	% !TEX root = ../agfa-master.tex
%% %%%%%%%%%%%%%%%%%%%%%%%%%%%%%%%
%% Abstract zu mwe-agfa.tex
%% %%%%%%%%%%%%%%%%%%%%%%%%%%%%%%%
\thispagestyle{empty} 
\section*{Danksagung}
Hier kann die Danksagung stehen und warum es dieses Arbeit gibt etc.
Aber: MAXIMAL EINE SEITE
\cleardoubleoddpage
			% Danksagung
	\tableofcontents 
	\thispagestyle{empty}
\endgroup

%% -- Zusammenfassung des Inhaltes
%% --
\cleardoubleoddpage 
	\pagenumbering{roman}
	% !TEX root = ../agfa-master.tex

%% %%%%%%%%%%%%%%%%%%%%%%%%%%%%%%%
%% ./0-AGFA-summary.tex
%% Zusammnfassung
%% Stand: 2022/10/10
%% %%%%%%%%%%%%%%%%%%%%%%%%%%%%%%%
\thispagestyle{empty} 
\addsec{Zusammenfassung}
Hier bitte eine kurze Zusammenfassung der Arbeit schreiben.
Diese soll nur die wesentliche Punkte erhalten.
Auf Definitionen, Sätze \etc kann verzichtet werden, da dies ja später kommt
\cleardoubleoddpage
		% Zusammenfassung
\cleardoubleoddpage 

%%\end{comment}

%% %%%%%%%%%%%%%%%%%%%%%%%%%
%% Mainmatter
%% %%%%%%%%%%%%%%%%%%%%%%%%

\pagenumbering{arabic}
\setcounter{page}{1}
\setcounter{section}{0}
\thispagestyle{empty}

%% -- Die einzelnen Abschnitte
%% -- Über \includeonly steuern, was aktuell bearbeitet wird
%% --

\include{./content/AGFA-Section-1}				% Erster Abschnitt
% !TEX root = ../AGFA-Master.tex
%% %%%%%%%%%%%%%%%%%%%%%%%%%%%%%%
%% ./content/AGFA-Section-2.tex
%% Stand: 2022/03/08
%% %%%%%%%%%%%%%%%%%%%%%%%%%%%%%% 
\thispagestyle{empty}
\section{Mein zweiter Abschnitt}
\subsection{Test der Listen}\label{subsec:listen}
%%
\begin{myenumerate}
\item
Aufzählung
\item
Aufzählung
\end{myenumerate}
%%
\begin{myequivalent}
\item
Äquivalent
\item
Äquivalent
\end{myequivalent}
%%
\begin{myitemize}
\item
Punkte
\item
Punkte
\end{myitemize}
%%
\begin{mynumber}
\item
Nummeriert
\item
Nummeriert
\end{mynumber}
%%
\subsection{Test der mathematischen Umgebungen}\label{subsec:mumgebungen}
Schon seit vielen hundert Jahren eines der schönsten Ergebnisse der Mathematik.
%
\begin{theorem}\label{thm:theorem1}
In einem rechtwinkligen Dreiecke mit den Seiten $ a $, $ b $ und der Hypothenuse $ c $ gilt stets
$	a^{ 2 } +b^{ 2 } = c^{ 2 } $.
\end{theorem}
%
\begin{proof}
Für den Beweis verweisen wir auf die Literatur, etwa \textcite{efhn:2016}
\end{proof}
%%
\begin{corollary}\label{cor:folgerung}
Hieraus folgt dann 
%
\[
	a^{ 2 } + b^{ 2 } = c^{ 2 } \, .
\]
%
\end{corollary}
%
\begin{proposition}\label{prop:prop}
Und nun ein kleiner Satz als Ergänzung 
\end{proposition}
%
\begin{lemma}
Zuvor aber ein Lemma
\end{lemma}
%
\begin{remark}
Eine Anmerkung
\end{remark}
%%
Mal sehen, wie die Eulersche Zahl und die imaginäre Einheit aussehen.
%%
\begin{theorem}\label{thm:eim}
$ \e^{ 2 \pi \im } = -1 $
\end{theorem}
%%
Alles andere kann jeder selbst mal testen.
%
\newpage
%%
\subsection{Querverweise}\label{subsec:referenzen}
\myquestion{Funktionieren alle Querverweise?}
%
Zunächst auf die Eulersche Zahl \vref{thm:eim} und dann auf Gleichung~\eqref{eq:pythagoras} in \vref{thm:theorem1}.
%
\subsection{Sinnvolle Literatur zu \LaTeX}
%
Mal ansehen: \textcite{l2tabu} und \textcite{lshort-german} \bzw \textcite{latex-refsheet} für all die Befehle und Möglichkeiten.
Wie man sieht, sind die Links auf die Dokumente hinterlegt.

				% Zweiter Abschnitt
% \include{name}									% etc

\cleardoubleoddpage 

%% %%%%%%%%%%%%%%%%%%%%%%%%
%% Backmatter
%% %%%%%%%%%%%%%%%%%%%%%%%%

\thispagestyle{empty}
\pagenumbering{roman}
\setcounter{page}{1}

%% -- Literaturverzeichnis
%%
%\nocite{}		% für Literatur, die nicht zitiert wurde; 
				% wenig sinnvoll dieses zu tun; siehe ReadMe.pdf
				
\RaggedRight		% Kein Blocksatz
\printbibliography

%% --

\end{document}
