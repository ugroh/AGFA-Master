 % !TEX TS-program = pdflatexmk
%% -------------------------- 
%% --  AGFA-Light
%% -- Stand:	2023/02/03   
%% -------------------------- 
\documentclass[%
	,ngerman
%	,fontsize		= 12pt
%	,DIV			= calc 
	,toc			= bib
	,abstract		= true	% oder auskommentieren
	,parskip			= half+
%	,BCOR 			= 10mm	% Bindekorrektur
	]{scrartcl}

%% -- Nutze die AGFA-Pakete
%% --
\usepackage[numeric]{./preamble/agfa-art}

%% -- Literatur - durch eigene erstezen 
%% --
\addbibresource{./bib/agfa-bib.bib}

%
\ExecuteBibliographyOptions{%
	,backref		= true	%
 	,url		= true		% online-Zitate separat 
 	,doi		= true		% true  DOI wird angezeigt
	,eprint		= false		% 
	}

%% -- Links farbig
%% --

\hypersetup{
	,colorlinks	= true    			%   onlineversion true                                                           
	,urlcolor	= blue       		%                                                              
	,citecolor	= blue      			%                                                          
	,linkcolor	= blue			 	% oder black
%  	,hidelinks 						% Vor dem Druck % entfernen
	}
	
%% -- Titelseite
%% --
\title{Die Vorlage \textsc{AGFA-Light.tex}}
\subtitle{-- Erläuterungen --}
\date{\today}
\author{U. Groh}

%% -- Inhaltsverzeichnis komprimiert
%% -- 

%% -- mit dotted line
\RedeclareSectionCommands[tocraggedpagenumber]{section,subsection} %% dotted line
%\RedeclareSectionCommands[tocraggedpagenumber,toclinefill={}]{section,subsection}
\BeforeStartingTOC[toc]{\begin{multicols}{2}}
\AfterStartingTOC[toc]{\end{multicols}}

%% -- Start des Dokuments
%% --	
\begin{document}

%% -- Titel  
%% --

%% \begin{comment}
\maketitle
\tableofcontents
%% \end{comment}

\thispagestyle{empty}

%% -- Dictum
%% -- 
\dictum[\href{https://de.wikipedia.org/wiki/Friedrich_Dürrenmatt}{F. Dürrenmatt}]{Das Rationale am Menschen sind seine Einsichten, das Irrationale, dass er nicht danach handelt.}

%% -- Zusammenfassung
%% --
\begin{abstract}
Dies ist eine kleine Übersicht zu \TeX{}, der Nutzung von \LaTeX{} und Erläuterungen zu den Vorlagen.
Dies ist \enquote{keine} fehlerfreie Ausarbeitung und sie kann nur für einen ersten Einstieg sinnvoll sein.
Gern helfe ich aber, wenn es Probleme damit oder es Fragen zur Nutzung von \LaTeX{} gibt.
\end{abstract}

%% -- Die Abschnitte
%% --
\input{./content/AGFA-Section-1}
% !TEX root = ../AGFA-Master.tex
%% %%%%%%%%%%%%%%%%%%%%%%%%%%%%%%
%% ./content/AGFA-Section-2.tex
%% Stand: 2022/03/08
%% %%%%%%%%%%%%%%%%%%%%%%%%%%%%%% 
\thispagestyle{empty}
\section{Mein zweiter Abschnitt}
\subsection{Test der Listen}\label{subsec:listen}
%%
\begin{myenumerate}
\item
Aufzählung
\item
Aufzählung
\end{myenumerate}
%%
\begin{myequivalent}
\item
Äquivalent
\item
Äquivalent
\end{myequivalent}
%%
\begin{myitemize}
\item
Punkte
\item
Punkte
\end{myitemize}
%%
\begin{mynumber}
\item
Nummeriert
\item
Nummeriert
\end{mynumber}
%%
\subsection{Test der mathematischen Umgebungen}\label{subsec:mumgebungen}
Schon seit vielen hundert Jahren eines der schönsten Ergebnisse der Mathematik.
%
\begin{theorem}\label{thm:theorem1}
In einem rechtwinkligen Dreiecke mit den Seiten $ a $, $ b $ und der Hypothenuse $ c $ gilt stets
$	a^{ 2 } +b^{ 2 } = c^{ 2 } $.
\end{theorem}
%
\begin{proof}
Für den Beweis verweisen wir auf die Literatur, etwa \textcite{efhn:2016}
\end{proof}
%%
\begin{corollary}\label{cor:folgerung}
Hieraus folgt dann 
%
\[
	a^{ 2 } + b^{ 2 } = c^{ 2 } \, .
\]
%
\end{corollary}
%
\begin{proposition}\label{prop:prop}
Und nun ein kleiner Satz als Ergänzung 
\end{proposition}
%
\begin{lemma}
Zuvor aber ein Lemma
\end{lemma}
%
\begin{remark}
Eine Anmerkung
\end{remark}
%%
Mal sehen, wie die Eulersche Zahl und die imaginäre Einheit aussehen.
%%
\begin{theorem}\label{thm:eim}
$ \e^{ 2 \pi \im } = -1 $
\end{theorem}
%%
Alles andere kann jeder selbst mal testen.
%
\newpage
%%
\subsection{Querverweise}\label{subsec:referenzen}
\myquestion{Funktionieren alle Querverweise?}
%
Zunächst auf die Eulersche Zahl \vref{thm:eim} und dann auf Gleichung~\eqref{eq:pythagoras} in \vref{thm:theorem1}.
%
\subsection{Sinnvolle Literatur zu \LaTeX}
%
Mal ansehen: \textcite{l2tabu} und \textcite{lshort-german} \bzw \textcite{latex-refsheet} für all die Befehle und Möglichkeiten.
Wie man sieht, sind die Links auf die Dokumente hinterlegt.



%% -- Literatur
%% --
\RaggedRight
% \setcounter{unbalance}{5}		
\nocite{voss:2012a,lamport:1986}
\printbibliography
%
\end{document}